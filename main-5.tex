\documentclass{article}
\newcommand\tab[1][1cm]{\hspace*{#1}}
\usepackage{fancyhdr}
\usepackage{enumerate}
\usepackage{enumitem}
\usepackage{amsmath}
\usepackage{amsfonts}
\usepackage{amssymb}
\usepackage{mathtools}
\usepackage{tikz}
\usetikzlibrary{calc}
\newcommand{\N}{\mathbb{N}}
\newcommand{\Z}{\mathbb{Z}}
\DeclarePairedDelimiter\ceil{\lceil}{\rceil}
\DeclarePairedDelimiter\floor{\lfloor}{\rfloor}
%
% Basic Document Settings
%

\topmargin=-0.45in
\evensidemargin=0in
\oddsidemargin=0in
\textwidth=6.5in
\textheight=9.0in
\headsep=0.25in

\linespread{1.1}

\pagestyle{fancy}
\lhead{\hmwkAuthorName}
\rhead{\hmwkClass: \hmwkTitle}
\cfoot{\thepage}

\renewcommand\headrulewidth{0.4pt}
\renewcommand\footrulewidth{0.4pt}

\setlength\parindent{0pt}

\newcommand{\hmwkTitle}{Homework\ \#4}
\newcommand{\hmwkDueDate}{November 8, 2017}
\newcommand{\hmwkClass}{CS 344}
\newcommand{\hmwkClassTime}{Section \#1}
\newcommand{\hmwkClassInstructor}{Professor Bahman Kalantari}
\newcommand{\hmwkAuthorName}{\textbf{Tina Janulis (trj31), Caleb Rodriguez (cjr199), and Rui Zhang (rz187)}}

%
% Title Page
%

\title{
    \vspace{2in}
    \textmd{\textbf{\hmwkClass:\ \hmwkTitle}}\\
    \normalsize\vspace{0.1in}\small{Due\ on\ \hmwkDueDate}\\
    \vspace{0.1in}\large{\textit{\hmwkClassInstructor\ \hmwkClassTime}}
    \vspace{3in}
}

\author{\hmwkAuthorName}
\date{}

\setlength\parskip{\baselineskip}
\begin{document}
\maketitle
\pagebreak

\section*{Problem 1}
\subsection*{Using shortest path algorithm determine of the following system of inequalities has a feasible solution.}
$x_4 - x_2 \leq −2$\\
$x_1 - x_3 \leq 3$\\
$x_5 - x_4  \leq -4$\\
$x_3 - x_2 \leq 2$\\
$x_2 - x_1 \leq 8$\\
$x_4 - x_3 \leq 1$\\
$x_1 - x_5 \leq −3$\\

If it has no solution can you change −2 in the first inequality to a number so that the system will be feasible? And what would be the smallest such value to change it to so that the system becomes feasible.

Based on these equations, we can determine the vertices and their distances in the initial graph:\\

\begin{center}
	\begin{tabular}{c c c c}
		\textbf{u} &\textbf{$->$} &\textbf{v} &\textbf{d(u,v)}\\
        $x_1$ &$->$& $x_2$ & 8\\
        $x_2$ &$->$& $x_3$ & 2\\
        $x_3$ &$->$& $x_4$ & 1\\
        $x_2$ &$->$& $x_4$ & -2\\
        $x_4$ &$->$& $x_5$ & -4\\
        $x_3$ &$->$& $x_1$ & 3\\
        $x_5$ &$->$& $x_1$ & -3\\
	\end{tabular}
    
\end{center}

From here, we can use Bellman-Ford to figure out if the system is feasible. Perform $n-1$ iterations of Bellman Ford where $n$ is the number of vertices. In this case, we know that there are 5 vertices $x_1$,$x_2$,$x_3$,$x_4$, and $x_5$. Therefore, we need to perform 4 iterations.\\

\textbf{First Iteration}
	\begin{center}
		\begin{tabular}{c c |c c | c c}
						&		&		&	 &\textbf{Final distances}\\        
			    $x_1:$&$0$ & $x_1:$&$0$ & $x_1:$&$-1$ \\
                $x_2:$&$\infty$ & $x_2:$&$8$ & $x_2:$&$8$ \\
                $x_3:$&$\infty$ & $x_3:$&$10$ & $x_3:$&$10$\\
                $x_4:$&$\infty$ & $x_4:$&$6$ & $x_4:$&$6$\\
                $x_5:$&$\infty$ & $x_5:$&$2$ & $x_5:$&$2$\\
		\end{tabular}
	\end{center}
    
\textbf{Second Iteration}
	\begin{center}
		\begin{tabular}{c c |c c}
						&		&		\textbf{Final distances}\\        
			    $x_1:$&$-1$ & $x_1:$&$-2$ \\
                $x_2:$&$8$  & $x_2:$&$7$ \\
                $x_3:$&$10$  & $x_3:$&$9$\\
                $x_4:$&$6$  & $x_4:$&$5$\\
                $x_5:$&$2$  & $x_5:$&$1$\\
		\end{tabular}
	\end{center}
    
 \textbf{Third Iteration}
	\begin{center}
		\begin{tabular}{c c |c c}
						&		&		\textbf{Final distances}\\        
			    $x_1:$&$-2$ & $x_1:$&$-3$ \\
                $x_2:$&$7$  & $x_2:$&$6$ \\
                $x_3:$&$9$  & $x_3:$&$8$\\
                $x_4:$&$5$  & $x_4:$&$4$\\
                $x_5:$&$1$  & $x_5:$&$0$\\
		\end{tabular}
	\end{center}
    
 \textbf{Fourth Iteration}
	\begin{center}
		\begin{tabular}{c c |c c}
						&		&		\textbf{Final distances}\\        
			    $x_1:$&$-3$ & $x_1:$&$-4$ \\
                $x_2:$&$6$  & $x_2:$&$5$ \\
                $x_3:$&$8$  & $x_3:$&$7$\\
                $x_4:$&$4$  & $x_4:$&$3$\\
                $x_5:$&$0$  & $x_5:$&$-1$\\
		\end{tabular}
	\end{center}
 We can conclude that the system is infeasible because there is a negative cycle with edges $x_1$ to $x_2$ to $x_4$ to $x_5$ back to $x_1$. The total sum of these edges is -1, so if we changed the -2 in the first equation to -1 ($x_4-x_2\leq-1$).
\section*{Problem 2}
\subsection*{Given 4 matrices, $A_1, . . . , A_4$, where $A_i$
is $m_{i - 1} × m_i$ matrix, $i = 1, . . . , 4$, with
$m_0 = 50, m_1 = 10, m_2 = 30, m_3 = 20, m_4 = 100$.
Compute $c(i, j)$'s by following the matrix chain multiplication algorithm}
\begin{center}
\begin{tabular}{c c c}
	$(A_1$x$A_2)$x$(A_3$x$A_4)$ & $(A_1$x$(A_2$x$A_3))$x$A_4$ & $A_1$x$((A_2$x$A_3)$x$A_4)$\\
    $A_1$x$A_2$ = 15000 mult. & $A_2$x$A_3$ = 6000 mult. & $A_2$x$A_3$ = 6000\\
    $A_3$x$A_4$ = 60000 mult. & x$A_1$ = 10000 mult. & x$A_4$ = 20000\\
    Last combination: 150000 mult. & x$A_4$ = 100000 mult. &x$A_1$ = 50000\\
    Total: 225000 mult. & Total: 116000 & Total: 76000\\
    \\
    & Optimal $C(i,j)$ = 76000
    
\end{tabular}
\end{center}
    
	
\section*{ Problem 3}
    

		

\end{document}


