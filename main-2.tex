\documentclass{article}
\newcommand\tab[1][1cm]{\hspace*{#1}}
\usepackage{fancyhdr}
\usepackage{enumerate}
\usepackage{enumitem}
\usepackage{amsmath}
\usepackage{amsfonts}
\usepackage{amssymb}
\usepackage{mathtools}
\newcommand{\N}{\mathbb{N}}
\newcommand{\Z}{\mathbb{Z}}
\DeclarePairedDelimiter\ceil{\lceil}{\rceil}
\DeclarePairedDelimiter\floor{\lfloor}{\rfloor}
%
% Basic Document Settings
%

\topmargin=-0.45in
\evensidemargin=0in
\oddsidemargin=0in
\textwidth=6.5in
\textheight=9.0in
\headsep=0.25in

\linespread{1.1}

\pagestyle{fancy}
\lhead{\hmwkAuthorName}
\rhead{\hmwkClass: \hmwkTitle}
\cfoot{\thepage}

\renewcommand\headrulewidth{0.4pt}
\renewcommand\footrulewidth{0.4pt}

\setlength\parindent{0pt}

\newcommand{\hmwkTitle}{Homework\ \#2}
\newcommand{\hmwkDueDate}{October 2, 2017}
\newcommand{\hmwkClass}{CS 344}
\newcommand{\hmwkClassTime}{Section \#1}
\newcommand{\hmwkClassInstructor}{Professor Bahman Kalantari}
\newcommand{\hmwkAuthorName}{\textbf{Tina Janulis (trj31), Caleb Rodriguez, and Rui Zhang (rz187)}}

%
% Title Page
%

\title{
    \vspace{2in}
    \textmd{\textbf{\hmwkClass:\ \hmwkTitle}}\\
    \normalsize\vspace{0.1in}\small{Due\ on\ \hmwkDueDate}\\
    \vspace{0.1in}\large{\textit{\hmwkClassInstructor\ \hmwkClassTime}}
    \vspace{3in}
}

\author{\hmwkAuthorName}
\date{}

\setlength\parskip{\baselineskip}
\begin{document}
\maketitle
\pagebreak

\section*{Problem 1}
\subsection*{Consider the binary numbers $x=11011010$ and $y = 10011011$. As in class, decomposing $x$ into $a$
, $b$ and $y$ into $c$, $d$, compute the product $xy$ using the divide-and-conquer method by forming $w1=a+b$,
$w2=c+d$, $u=w1 w2$, $v=ac$, $w=bd$. Recall $xy=2^nv+2^{n/2}(u-v-w)+w$. Each multiplication by $2$
amounts to a shift.}
\begin{center}
	\begin{tabular} {c c c c c}
        We know that $x=11011010$ and $y=10011011$.\\
        Let us separate these into 4-bit strings such that $a=1101$, $b=1010$, $c=1001$, and $d=1011$.\\
        Furthermore, we know that $x$ and $y$ are both 8-bit, so $n=8$.\\
        Next, we convert each one of these 4-bit numbers into decimal numbers.\\
        Thus, we can say $a=13$, $b=10$, $c=9$, $d=11$.\\
        Next, we introduce variables $w1$, $w2$, $v$, $u$, and $w$.\\
        We let $w1=a+b=23$, $w2=c+d=20$, $u=w1 w2=ac+ad+bc+bd=460$.\\
        Next, we let $v=ac=117$ and $w=bd=110$.\\
        Plugging this into the longer equation, we get $xy=(2^8)(117)+(460-117-110)(2^4)+110$.\\
        $=(256)(117)+(233)(16)+110$\\
        $=29952+3728+110$\\
        $=33790$\\
        As a bit string, this could be represented as $1000001111111110$.\\
	\end{tabular}
\end{center}

\section*{Problem 2}
\subsection*{You are given an infinite array $A[i]$ in which the first $n$ elements are integers in sorted order and the
rest are filled with $∞$, You are not given $n$. Describe an algorithm that takes as input an integer $x$ and finds
a position in the array containing $x$, if such a position exists, in $O(log n)$ time.}
	\begin{center}
        \begin{enumerate}
			\item Take an array $A[i]$ with infinite size.
            \item Let $right=1$, $left=1$, and $mid=0$.
            \item While($A[right] \neq ∞ $).
            \item 	$left=right$
            \item 	$right=right*2$
            \item endWhile
            \item While($left-right>1$)
            \item 	$mid=left+\floor*{\frac{x}{2}}$.
            \item 	if($A[mid]=∞$)
            \item 		$right=mid$
            \item	else
            \item		$left=mid$
            \item endWhile
            \item $n=left+1$			//$n$ is the size of the integer portion of the array.	
            \item Call $binarySearch(A, n, x)$
            \item if($x$ is found)
            \item	return index of $x$
            \item else
            \item 	return $∞$
            \item end
        \end{enumerate}
	\end{center}
\section*{Problem 3}
	\subsection*{Solve each of the recurrence relations and give $\Theta$ bound for each. You can use the master theorem if applicable.}
	\begin{enumerate}[label=(\alph*)]
		\item $T(n) = 5$T$(n/4)+n$
        \begin{center}
        	\begin{tabular}{c c c c c}
        	Let $n=4^k$ where $k=\log_{4}n$. Let's apply this to the equation:\\
            $T(4^k)$ &$=$ &$5T(4^{n-1})+4^n$\\
            &$=$ &$5(5T(4^{k-2})+4^{k-1})+4^k$\\
           	&$=$ &$5^2T(4^{k-2})+5*4^{k-1}+4^k$\\
            &$=$ &$5^2T(4{k-2})+5*4^k(4^{-1}+1)$\\
            &$=$ &$5^2(5T(4^{k-3}+4^{k-2})+4^k(4^{-1}+1)$\\
            &$=$ &$5^3T(4^{k-3})+4^{k}(5^{-2}+5^{-1}+1)$\\
            As $k$ approaches $\infty$, we can conclude that:\\
           	$T(4^k)=5^kT(4^0)+4^k*\sum_{i=1}^{n-1} (4/5)^i$.\\
            This then translates to:\\
            $T(4^k)=0+4^k\frac{5/4^{k-1}}{5/4-1})$\\
            And finally:\\
            $n\Theta(5^k)=n\Theta(n\log_{4}5)=\Theta(n\log_{4}5)$
       	\end{tabular}
        \end{center}   	
        \item $T(n) = 7T(n/7)+n$
        \begin{center}
        	\begin{tabular}{c c c c c}
        	Let $n=7^k$ where $k=\log_{7}n$. Let's apply this to the equation:\\
            $T(7^k)$ &$=$ &$7T(7^{k-1})+7^k$\\
			&$=$ &$7*7(T(7^{k-2})+7^{k-1})+7^k$\\
            &$=$ &$7^2T(7^{k-2})+2(7^{k})$\\
            &$=$ &$7^3T(7^{k-3})+3(7^{k})$\\
            As $k$ approaches $\infty$, we can conclude that:\\
            $T(7^k)=7^k(7^0)+k7^k=k7^k=n\log_7{n}$\\
            Finally, this means:\\
            $\Theta(n\log_7{n})$\\
			\end{tabular}
        \end{center}
        \item $T(n)=9T(n/4)+n^2$
        	\begin{center}
        		\begin{tabular}{c c c c c}
        		Let $n=4^k$ where $k=\log_{4}n$. Let's apply this to the equation:\\
                $T(4^k)$ &$=$ &$9T(4^{k-1})+4^{2k}$\\
                &$=$ &$9*9(T(4^{k-2})+9*4^{2k-2})+4^{2k}$\\
                &$=$ &$9^2T(4^{k-2})+4^{2k}(9*4^{-2}+1)$\\
                &$=$ &$9^3T(4^{k-3})+9^2*4^{2k-6}+4^{2k}(9*4^{-2}+1)$\\
            	&$=$ &$9^3T(4^{k-3})+4^{2k}(9^2*4^{-3}+9*4^{-2}+1)$\\
               	As $k$ approaches $\infty$, we can conclude that:\\
				$T(4^k)=9^k(T(4^0))+\sum_{i=1}^{k-1}(9/4^{2})^i$\\
				Furthermore:\\
                $T(4^k)=0+n^2((9/16)^{k-1}-1)=\Theta(n^2)$.
        		\end{tabular}
        	\end{center}
           \item $T(n)=8T(n/2)+n^3$
			\begin{center}
				\begin{tabular}{c c c c c}
				  Let $n=2^k$ where $k=\log_{2}n$. Let's apply this to the equation:\\
                  $T(2^k)$ &$=$ &$8T(2^{k-1})+2^{3k}$\\
                  &$=$ &$8T(2^{k-1})+8^k$\\
                  &$=$ &$8^2T(2^{k-2})+2*8^k$\\
                  As $k$ approaches $\infty$, we can conclude that:\\
                  $T(2^k)=8^k(T(2^0))+k8^k$\\
                  Therefore:\\
                  $\Theta(n\log_{2}n)$.
				\end{tabular}
			\end{center}
            
            \item $T(n)=49T(n/25)+n^3.5\log{n}$
            	\begin{center}
            		\begin{tabular}{c c c c c}
            		 Let $n=25^k$ where $k=\log_{25}n$. Let's apply this to the equation:\\
					$T(25^n)$ $=$ $49T(25^{k-1})+25^{3.5}\log{25^{k}}$\\
                    $=$ $49^2T(25^{k-2})+49*25^{3.5(k-1)}\log{{25}^{k-1}}+25^{3.5k}\log{25^{k}}$\\
                    $=$ $49^2T(25^{k-1})+25^{3.5k}log{25^k}(49*25^{-3.5}log{25^{-1}}+1)$\\
                    As $k$ approaches $\infty$, we can conclude that:\\
                    $T(25^n)$ $=$ $49^kT(1)+n^{3.5}\log{n}\sum{i=1}^{k-1}{\frac{49\log{1/25^i}}{25^i}}$
                    Therefore:\\
                    $\Theta(n^{log_{25}{49}})$
            		\end{tabular}
            	\end{center}
          \item $T(n)=T(\sqrt[]{n})+1$
          	\begin{center}
          		\begin{tabular}{c c c c c}
          		Let $n=2^k$ where $k=log_{2}n$. Let's apply this to the equation:\\
                $T(2^k)=T(2^{k/2})+1$\\
                Let $S(n)=T(2^{k}$.\\
                This implies that $2T(2^{k/2})=S(k/2)
                $S(k)=2S(k/2)+1\\
                 As $k$ approaches $\infty$, we can conclude that:\\
                 $S(k)=\Theta(\log{k})=\Theta(\log\log{n})$
          		\end{tabular}
          	\end{center}
            \item $T(n)=T(\sqrt[]{n})+\log{n}$
            	\begin{center}
            		\begin{tabular}{c c c c c}
            			$T(2^k)=T(2^{k/2})+\log{k/2}$\\
                        Let $S(k)=T(2^k)$\\
                       $S(k)=2S(k/2)+k$\\
                       $S(k)=\Theta(k\log{k})$
                       Therefore:\\
                       $\Theta(\log{n}\log{\log{n}})$\\
            		\end{tabular}
            	\end{center}
\end{enumerate}
\section*{Problem 4}
Find coefficient of polynomial of deg. 2, \(p(x) = a_2x^2+a_1x+a_0\), such that p(1) = -1, p(2) = 1, p(3) = 0.
\begin{align*}
p(1) &\rightarrow a_2 + a_1 + a_0 = -1 \\
p(2) &\rightarrow 4a_2 + 2a_1 + a_0 = 1 \\
p(3) &\rightarrow 9a_2 + 6a_1 + a_0 = 0 \\
\text{Converting to matrix} \\
\begin{pmatrix}
1 & 1 & 1 & -1 \\
4 & 2 & 1 & 1 \\
9 & 3 & 1 & 0 \\
\end{pmatrix} \\
\text{Performing RREF on the matrix would yield} \\
a_0 &= -6 \\
a_1 &= \frac{13}{2} \\
a_2 &= -\frac{3}{2}
\end{align*}
\section*{Problem 5}
\begin{align}
25&, 14, 63, 29, 63, 47, 12, 21 \text{ Start} \\
25 (piv) &, 14 (<), 63 (>), 29 (>), 63 (>), 47 (>), 12 (<), 21 (<) \\
25 (piv) &, 14 (<), 12 (<), 21 (<), 63 (>), 47 (>), 63 (>), 29 (>) \\
14 &, 12, 21, 25, 63, 47, 63, 29 \\
14 (piv) &, 12 (<), 21 (>), 25 (>), 63 (>), 47 (>), 63 (>), 29 (>) \\
12 &, 14, 21, 25, 63, 47, 63, 29 \\
12 &, 14, 21, 25, 63 (piv), 47 (<), 63 (>), 29 (<) \\
12 &, 14, 21, 25, 47, 29, 63, 63 \\
12 &, 14, 21, 25, 47 (piv), 29 (<), 63 (>), 63 (>) \\
12 &, 14, 21, 25, 29, 47, 63, 63 \\
12 &, 14, 21, 25, 29, 47, 63 (piv), 63 (>) \\
12 &, 14, 21, 25, 29, 47, 63, 63
\end{align}
\end{document}