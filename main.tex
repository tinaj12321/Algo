\documentclass{article}
\newcommand\tab[1][1cm]{\hspace*{#1}}
\usepackage{fancyhdr}
\usepackage{enumerate}
\usepackage{enumitem}
\usepackage{amsmath}
\usepackage{amsfonts}
\usepackage{amssymb}
\newcommand{\N}{\mathbb{N}}
\newcommand{\Z}{\mathbb{Z}}
%
% Basic Document Settings
%

\topmargin=-0.45in
\evensidemargin=0in
\oddsidemargin=0in
\textwidth=6.5in
\textheight=9.0in
\headsep=0.25in

\linespread{1.1}

\pagestyle{fancy}
\lhead{\hmwkAuthorName}
\rhead{\hmwkClass: \hmwkTitle}
\cfoot{\thepage}

\renewcommand\headrulewidth{0.4pt}
\renewcommand\footrulewidth{0.4pt}

\setlength\parindent{0pt}

\newcommand{\hmwkTitle}{Homework\ \#1}
\newcommand{\hmwkDueDate}{September 20, 2017}
\newcommand{\hmwkClass}{CS 344}
\newcommand{\hmwkClassTime}{Section \#1}
\newcommand{\hmwkClassInstructor}{Professor Bahman Kalantari}
\newcommand{\hmwkAuthorName}{\textbf{Tina Janulis, Caleb Rodriguez, and Ray Zhang}}

%
% Title Page
%

\title{
    \vspace{2in}
    \textmd{\textbf{\hmwkClass:\ \hmwkTitle}}\\
    \normalsize\vspace{0.1in}\small{Due\ on\ \hmwkDueDate}\\
    \vspace{0.1in}\large{\textit{\hmwkClassInstructor\ \hmwkClassTime}}
    \vspace{3in}
}

\author{\hmwkAuthorName}
\date{}

\setlength\parskip{\baselineskip}
\begin{document}
\maketitle
\pagebreak

\section*{Problem 1}
\subsection*{In each of the following situations, indicate whether $f=O(g)$, or $f=\Omega(g)$, or both (in which case $f=\Theta(g)$).}
\begin{enumerate}[label=(\alph*)]
\item \addtocounter{enumi}{1}$f(n)=n-100$ 
\\$g(n)=n-200$
	\begin{center}
    	\begin{tabular} {c c c c c}
    		When $c_1=1$, $c_2=100$ and $n=300$, $c_1g(n)\leq f(n)\leq c_2g(n)$\\
            Therefore, $f(n)=\Theta(g(n))$\\
        \end{tabular}
    \end{center}
\item \addtocounter{enumi}{1} $f(n)=100n+$log$n$
\\ $g(n)=n+($log$n)^2$
	\begin{center}
    	\begin{tabular} {c c c c c}
        	When $c_1=100$, $c_2=200$ and $n=1$, $c_1g(n)\leq f(n)\leq c_2g(n)$\\
            Therefore, $f(n)=\Theta(g(n))$\\
        \end{tabular}
    \end{center}
\item \addtocounter{enumi}{1} $f(n)=$log$2n$
\\ $g(n)=$log$3n$
	\begin{center}
    	\begin{tabular} {c c c c c}
        	When $c_1=1$, $c_2=200$ and $n=3$, $c_1g(n)\leq f(n)\leq c_2g(n)$\\
            Therefore, $f(n)=\Theta(g(n))$\\
        \end{tabular}
    \end{center}
\item \addtocounter{enumi}{1} $f(n)=n^{1.01}$
\\ $g(n)=n$log$^2n$
	\begin{center}
    	\begin{tabular} {c c c c c}
        	When $c=2$ and $n=3$, $f(n)\leq cg(n)$\\
            Therefore, $f(n)=\Omega(g(n))$\\
        \end{tabular}
    \end{center}

\item \addtocounter{enumi}{1} $f(n)=n^{.01}$
\\ $g(n)=n$log$^2n$
	\begin{center}
    	\begin{tabular} {c c c c c}
        	When $c=2$ and $n=100$, $f(n)\leq cg(n)$\\
            Therefore, $f(n)=\Omega(g(n))$\\
        \end{tabular}
	\end{center}
\item \addtocounter{enumi}{1}  $f(n)=n^{.01}$
\\ $g(n)=n$log$^2n$
	\begin{center}
    	\begin{tabular} {c c c c c}
        	When $c=2$ and $n=30$, $f(n)\leq cg(n)$\\
            Therefore, $f(n)=\Omega(g(n))$\\
        \end{tabular}
	\end{center}

\item \addtocounter{enumi}{1}$f(n)=n2^{n}$
\\ $g(n)=3^{n}$
	\begin{center}
    	\begin{tabular} {c c c c c}
        	When $c=200$ and $n=1$, $f(n)\leq cg(n)$\\
            Therefore, $f(n)=\Omega(g(n))$\\
        \end{tabular}
	\end{center}

\item \addtocounter{enumi}{1}$f(n)=n!$
\\ $g(n)=2^n$
	\begin{center}
    	\begin{tabular} {c c c c c}
        	When $c=1$ and $n=200$, $f(n)\leq cg(n)$\\
            Therefore, $f(n)=\Omega(g(n))$\\
        \end{tabular}
	\end{center}
\item \addtocounter{enumi}{1} $f(n)=\sum_{i=1}^{n} i^{k}$
\\ $g(n)=n^{k+1}$
	\begin{center}
    	\begin{tabular} {c c c c c}
        	$\sum_{i=1}^{n}=n(n^{k})$\\
            This could then be translated into $f(n)=n^{k+1}$.\\
            This is equivalent to $g(n)=n^{k+1}$
            Therefore, $f(n)=\Theta(g(n))$\\
        \end{tabular}
	\end{center}
\end{enumerate}

\section*{Problem 2}
\subsection*{Show that, if $c$ is a positive real number, then $g(n)=1+c^1+c^2+...+c^{n}$ is:}
\begin{enumerate}[label=(\alph*)]
\item $\Theta(1)$ if $c<1$.
	\begin{center}
    	\begin{tabular}{c c c c c}
        	If $0<c<1$ and its exponent $n$ is increasing, $c^n$ will decrease as $n$ increases.\\
            Essentially, $ \lim_{x \to 0} g(n)=-\infty$.\\
            Therefore, 1 is the leading number and $g(n)=\Theta(1)$.\\
        \end{tabular}
	\end{center}
\item $\Theta(n)$ if $c=1$.
	\begin{center}
    	\begin{tabular}{c c c c c}
        	If $c=1$, 1 is being added to itself $n$ times, which translates to\\
            $\sum_{i=1}^{n} 1^{n}=n$.\\
            Therefore, n is the leading coefficient and $g(n)=\Theta(n)$.\\
        \end{tabular}
	\end{center}
\item $\Theta(c^n)$ if $c>1$.
\begin{center}
	\begin{tabular}{c c c c c}
    	If $c>1$, $c^n$ will keep increasing until the last $c^n$ which will be the highest number in the series.\\
        Essentially, $ \lim_{x \to 0} g(n)=\infty$.\\
        Therefore, $c^n$ is the leading number and $g(n)=\Theta(c^n)$.\\
	\end{tabular}
\end{center}
\end{enumerate}
\section*{Problem 3}
\subsection*{The Fibonacci numbers $F_0,F_1,F_2,...,$ are defined by the rule\\
	\begin{center}
		\begin{tabular}{c c c c c}
        $F_0=0,F_1=1,F_{n}=F_{n-1}+F_{n-2}$
    	\end{tabular}
	\end{center}
In this problem we will confirm that this sequence grows exponentially fast and obtain some bounds on its growth.\\}
\begin{enumerate}[label=(\alph*)]
	\item After considering the function, say $n=6$. Then, for the basic step we can say:
    \begin{center}
		\begin{tabular}{c c c c c}
        $F_6=F_5+F_4$\\
        $F_6=5+3$\\
        $F_6=8$\\
    	\end{tabular}
	\end{center}
    With this in mind, we can prove that $F_n\geq2 ^{0.5n}$ for $n=6$:
    \begin{center}
    	\begin{tabular}{c c c c c}
    	$F_6\geq 2^{0.5(6)}$\\
        $8\geq 2^3$\\
        $8\geq 8$ which is true.\\
    	\end{tabular}
    \end{center}
    For the inductive step we can first show that when $n=n+1$:
    \begin{center}
    	\begin{tabular}{c c c c c}
    	$F_{n+1}=F_n+F_{n-1}$
    	\end{tabular}
    \end{center}
    And based on the basic step, we can start proving $F_n\geq2 ^{0.5n}$ inductively:
    \begin{center}
    	\begin{tabular} {c c c c c}
        	$F_n+F_{n-1}$ &$\geq$ &$2^{0.5(n-1)}+2^{0.5n}$\\
            &$\geq$ &$2^{0.5(n-1)}(\sqrt[]{2}+1)$\\
            &$\geq$ &$2*2^{0.5(n-1)}$\\
            &$\geq$ &$2^{0.5n-0.5+1}$\\
            &$\geq$ &$2^{0.5n+0.5}$\\
            &$\geq$ &$2^{0.5(n+1)}$\\
        \end{tabular}
    \end{center}
    Based on the given equation, $F_{n}=F_{n-1}+F_{n-2}$, we can conclude that $F_n+F_{n-1}=F{n+1}$.\\
    We are trying to prove that $F_n \geq2 ^{0.5n}$.\\
    The inductive proof above can translate to:
    \begin{center}
    	\begin{tabular}{c c c c c}
    	$F_{n+1}$ &$\geq$ &$2^{0.5(n+1)}$\\
    	\end{tabular}
    \end{center}
    Which proves, inductively, that $F_n\geq2 ^{0.5n}$.\\
	\item By guess and check, you have to select a value for $c$ and $n$ that makes $1\leq 2^{cn}$.\\
    \begin{center}
    	\begin{tabular}{c c c c c}
    	When $c=0.9$ and $n=3$, $2^{cn}=2^{2.7}$.\\
        With these values for $c$ and $n$, $1\leq2^{2.7}$.\\
    	\end{tabular}
    \end{center}
	\item For \(F_n = \Omega(2^{cm})\), there must be some $k$ where \(F_n \geq k(2^{cn}\) where $k = 2^x$. Therefore,
    \begin{align}
    	F_n &= F_{n-1} + F_{n-2} \\
        F_n &\geq 2^{c(n-1)+x} + 2^{c(n-2)+x} \\
        F_n &\geq 2^{cn+m}(2^{-c}+2^{-2c})
  \end{align}
  since \(F_n \geq 2^{cn+m}\), then it must be true that \(2^{-c}+2^{-2c} \geq 1\) is also true if (3) is to remain true. Solving \(2^{-c}+2^{-2c} = 1\) yields $c \approx 0.694$.
\end{enumerate}
\section*{Problem 4}
\subsection*{ Is there a faster way to compute the nth Fibonacci number than by fib2 (page 13)? One idea involves matrices.We start by writing the equations and  in matrix notation:}
	\begin{center}
			$\begin{bmatrix}
				 F_{1}\\
				 F_{2}\\
				 
			\end{bmatrix}$
            =
            $\begin{bmatrix}
				 0&1\\
				 1&1\\ 
			\end{bmatrix}$
            .
            $\begin{bmatrix}
				 F_{0}\\
				 F_{1}\\ 
			\end{bmatrix}$
	\end{center}
Similarly, 
\begin{center}
			$\begin{bmatrix}
				 F_{1}\\
				 F_{2}\\
				 
			\end{bmatrix}$
            =
            $\begin{bmatrix}
				 0&1\\
				 1&1\\ 
			\end{bmatrix}$
            .
            $\begin{bmatrix}
				 F_{0}\\
				 F_{1}\\ 
			\end{bmatrix}$
            $=$
            $\begin{bmatrix}
           		 0&1\\
				 1&1\\
            \end{bmatrix}^2$
           .
           $\begin{bmatrix}
				 F_{0}\\
				 F_{1}\\ 
			\end{bmatrix}$
	\end{center}
   and in general,
   \begin{center}
   	$\begin{bmatrix}
				 F_{n}\\
				 F_{n+1}\\ 
	\end{bmatrix}$
    =
     $\begin{bmatrix}
           		 0&1\\
				 1&1\\
            \end{bmatrix}^n$
            .
       $\begin{bmatrix}
				 F_{0}\\
				 F_{1}\\ 
			\end{bmatrix}$     
   \end{center}
\begin{enumerate}[label=(\alph*)]
\item So, in order to compute $F_n$, it suffices to raise this $2\times2$ matrix, call it $X$, to the $n$th power.
Show that two $2\times2$ matrices can be multiplied using 4 additions and 8 multiplications. But how many matrix multiplications does it take to compute $X_n$?
\begin{center}
    Take two $2\times2$ matrices:
    $A=$
    $\begin{bmatrix}
    1&2\\
    3&4\\
    \end{bmatrix}$
    and $B=$
    $\begin{bmatrix}
    5&6\\
    7&8\\
    \end{bmatrix}$.
   \end{center}
   
   \begin{center}
   \begin{tabular} {c c c c c}
    When multiplying these two matrices you wind up with the following matrix:\\
    \end{tabular}
    $\begin{bmatrix}
    1(5)+2(7)&1(6)+2(8)\\
    3(5)+4(7)&3(6)+4(8)\\
    \end{bmatrix}$
    \begin{tabular}{c c c c c}
    Based on this operation, you can see that for a $n\times n$ matrix \\
    there are $2n$ additions and $4n$ multiplications.\\
    Essentially, the amount of multiplications is twice the amount of additions \\
    and the amount of additions is twice the amount of elements in a row or column for a $n\times n$ matrix.\\
    \end{tabular}
    \end{center}
  
\item Show that $O$(log$n$) matrix multiplications suffice for computing $X^n$.

Assume that we have to solve \(X^{2n}\) where $n \geq 1$. Then, the following process could occur
\begin{align}
	x_i(x_i) = x_i^2 &= X_{1i} \\
    x_i^4 = X_{1i}(X_{1i}) &= X_{2i} \\
    x_i^8 = X_{2i}(X_{2i}) &= X_{3i} \\
    etc.
  \end{align}
This process doubles the exponent for each iteration, and in the example above, $X^8$ is done in 3 steps against 8.
\end{enumerate}
\end{document}
